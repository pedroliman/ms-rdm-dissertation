\documentclass[]{elsarticle} %review=doublespace preprint=single 5p=2 column
%%% Begin My package additions %%%%%%%%%%%%%%%%%%%
\usepackage[hyphens]{url}
\usepackage{lineno} % add
\providecommand{\tightlist}{%
  \setlength{\itemsep}{0pt}\setlength{\parskip}{0pt}}

\bibliographystyle{elsarticle-harv}
\biboptions{sort&compress} % For natbib
\usepackage{graphicx}
\usepackage{booktabs} % book-quality tables
%% Redefines the elsarticle footer
%\makeatletter
%\def\ps@pprintTitle{%
% \let\@oddhead\@empty
% \let\@evenhead\@empty
% \def\@oddfoot{\it \hfill\today}%
% \let\@evenfoot\@oddfoot}
%\makeatother

% A modified page layout
\textwidth 6.75in
\oddsidemargin -0.15in
\evensidemargin -0.15in
\textheight 9in
\topmargin -0.5in
%%%%%%%%%%%%%%%% end my additions to header

\usepackage[T1]{fontenc}
\usepackage{lmodern}
\usepackage{amssymb,amsmath}
\usepackage{ifxetex,ifluatex}
\usepackage{fixltx2e} % provides \textsubscript
% use upquote if available, for straight quotes in verbatim environments
\IfFileExists{upquote.sty}{\usepackage{upquote}}{}
\ifnum 0\ifxetex 1\fi\ifluatex 1\fi=0 % if pdftex
  \usepackage[utf8]{inputenc}
\else % if luatex or xelatex
  \usepackage{fontspec}
  \ifxetex
    \usepackage{xltxtra,xunicode}
  \fi
  \defaultfontfeatures{Mapping=tex-text,Scale=MatchLowercase}
  \newcommand{\euro}{€}
\fi
% use microtype if available
\IfFileExists{microtype.sty}{\usepackage{microtype}}{}
\ifxetex
  \usepackage[setpagesize=false, % page size defined by xetex
              unicode=false, % unicode breaks when used with xetex
              xetex]{hyperref}
\else
  \usepackage[unicode=true]{hyperref}
\fi
\hypersetup{breaklinks=true,
            bookmarks=true,
            pdfauthor={},
            pdftitle={Análise RDM Piloto - Modelo de Bass},
            colorlinks=true,
            urlcolor=blue,
            linkcolor=magenta,
            pdfborder={0 0 0}}
\urlstyle{same}  % don't use monospace font for urls
\setlength{\parindent}{0pt}
\setlength{\parskip}{6pt plus 2pt minus 1pt}
\setlength{\emergencystretch}{3em}  % prevent overfull lines
\setcounter{secnumdepth}{0}
% Pandoc toggle for numbering sections (defaults to be off)
\setcounter{secnumdepth}{0}
% Pandoc header


\usepackage[nomarkers]{endfloat}

\begin{document}
\begin{frontmatter}

  \title{Análise RDM Piloto - Modelo de Bass}
    \author[GMAP \textbar{} UNISINOS]{Pedro Nascimento de Lima\corref{c1}}
   \ead{pedro@example.com} 
   \cortext[c1]{Corresponding Author}
    \author[Another University]{Bob Security}
   \ead{bob@example.com} 
  
      \address[Some Institute of Technology]{Department, Street, City, State, Zip}
    \address[Another University]{Department, Street, City, State, Zip}
  
  \begin{abstract}
  Esta é uma análise piloto utilizando o RDM. Estou usando este documento
  nesta formatação para integrar os resultados do R dentro do documento
  sem precisar copiar e colar. No final da análise este texto vai todo
  para o word. Este documento e análise servirão como um template para a
  análise ``real'' do RDM a ser integrada na dissertação, porém deve
  conter todos os elementos de uma análise RDM.
  \end{abstract}
  
 \end{frontmatter}

\section{Análise RDM Piloto - O Modelo de
Bass}\label{analise-rdm-piloto---o-modelo-de-bass}

\paragraph{Propósito}\label{proposito}

Este documento apresenta uma análise RDM piloto do modelo de Bass. O
objetivo desta análise foi exercitar a aplicação do RDM com um exemplo
conhecido pela literatura atual, com o propósito de desenvolver a
competência de analisar modelos com o RDM. Neste processo, foi
necessário desenvolver algoritmos para a execução e análise dos
experimentos computacionais.

O modelo de Bass (1969) é amplamente reconhecido na literatura
(frequentemente citado entre os 10 trabalhos mais influentes nos
periódicos da INFORMS, e utilizado por Sterman (2000) e Morecroft (2001)
em seus livros a respeito de dinâmica de sistemas).

A análise do modelo de Bass é propícia para este propósito por dois
motivos. Primeiro, o modelo representa um dos fatores mais incertos para
as empresas privadas: Como será a evolução da demanda de um novo
produto. Este modelo foi aplicado em diversas situações, e é capaz de
representar processos de crescimento da demanda em novos produtos.
Segundo, a execução do modelo depende de parâmetros altamente incertos
(ex.: A probabilidade de que um consumidor atual do produto divulgue o
produto a outro consumidor).

O Modelo de Bass representa o processo de adoção de novos produtos, e
propõe-se a identificar / predizer variáveis importantes para a empresa
(ex.: qual será e quando será o pico de vendas de um novo produto?).

\paragraph{Usage}\label{usage}

Once the package is properly installed, you can use the document class
\emph{elsarticle} to create a manuscript. Please make sure that your
manuscript follows the guidelines in the Guide for Authors of the
relevant journal. It is not necessary to typeset your manuscript in
exactly the same way as an article, unless you are submitting to a
camera-ready copy (CRC) journal.

\paragraph{Functionality}\label{functionality}

The Elsevier article class is based on the standard article class and
supports almost all of the functionality of that class. In addition, it
features commands and options to format the

\begin{itemize}
\item
  document style
\item
  baselineskip
\item
  front matter
\item
  keywords and MSC codes
\item
  theorems, definitions and proofs
\item
  lables of enumerations
\item
  citation style and labeling.
\end{itemize}

\section{Front matter}\label{front-matter}

The author names and affiliations could be formatted in two ways:

\begin{enumerate}
\def\labelenumi{(\arabic{enumi})}
\item
  Group the authors per affiliation.
\item
  Use footnotes to indicate the affiliations.
\end{enumerate}

See the front matter of this document for examples. You are recommended
to conform your choice to the journal you are submitting to.

\section{Bibliography styles}\label{bibliography-styles}

There are various bibliography styles available. You can select the
style of your choice in the preamble of this document. These styles are
Elsevier styles based on standard styles like Harvard and Vancouver.
Please use BibTeX~to generate your bibliography and include DOIs
whenever available.

Here are two sample references: Feynman and Vernon Jr. (1963; Dirac
1953).

\section*{References}\label{references}
\addcontentsline{toc}{section}{References}

\hypertarget{refs}{}
\hypertarget{ref-Dirac1953888}{}
Dirac, P.A.M. 1953. ``The Lorentz Transformation and Absolute Time.''
\emph{Physica} 19 (1---12): 888--96.
doi:\href{https://doi.org/10.1016/S0031-8914(53)80099-6}{10.1016/S0031-8914(53)80099-6}.

\hypertarget{ref-Feynman1963118}{}
Feynman, R.P, and F.L Vernon Jr. 1963. ``The Theory of a General Quantum
System Interacting with a Linear Dissipative System.'' \emph{Annals of
Physics} 24: 118--73.
doi:\href{https://doi.org/10.1016/0003-4916(63)90068-X}{10.1016/0003-4916(63)90068-X}.

\end{document}


