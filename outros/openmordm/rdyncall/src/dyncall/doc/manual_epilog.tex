%//////////////////////////////////////////////////////////////////////////////
%
% Copyright (c) 2007,2009 Daniel Adler <dadler@uni-goettingen.de>, 
%                         Tassilo Philipp <tphilipp@potion-studios.com>
%
% Permission to use, copy, modify, and distribute this software for any
% purpose with or without fee is hereby granted, provided that the above
% copyright notice and this permission notice appear in all copies.
%
% THE SOFTWARE IS PROVIDED "AS IS" AND THE AUTHOR DISCLAIMS ALL WARRANTIES
% WITH REGARD TO THIS SOFTWARE INCLUDING ALL IMPLIED WARRANTIES OF
% MERCHANTABILITY AND FITNESS. IN NO EVENT SHALL THE AUTHOR BE LIABLE FOR
% ANY SPECIAL, DIRECT, INDIRECT, OR CONSEQUENTIAL DAMAGES OR ANY DAMAGES
% WHATSOEVER RESULTING FROM LOSS OF USE, DATA OR PROFITS, WHETHER IN AN
% ACTION OF CONTRACT, NEGLIGENCE OR OTHER TORTIOUS ACTION, ARISING OUT OF
% OR IN CONNECTION WITH THE USE OR PERFORMANCE OF THIS SOFTWARE.
%
%//////////////////////////////////////////////////////////////////////////////

\newpage

\section{Epilog}

\subsection{Stability and security considerations}

Since the \product{dyncall} library doesn't know anything about the called
function itself (except its address), no parameter-type validation is done.
Thus in order to avoid crashes, data corruption, etc., the user is urged
to ascertain the number and types of parameters. It is strongly advised to
double check the parameter types of every function to be called, and not to
call unknown functions at all.\\

Consider a simple program that issues a call by directly passing some
command line arguments to the call itself, or even worse, by indirectly
choosing a library and a function to call. Such unchecked input data can be
quite easily used to intentionally crash the program , or to hijack it and
take control of the program flow.\\
To put it in a nutshell, if not used with care, programs depending on the
\product{dyncall}, \product{dyncallback} and \product{dynload} libraries,
can become arbitrary function call dispatchers by manipulating their input
data. Successful exploits of programs like the one outlined above can be
misused as very powerful tools for a wide variety of malicious attacks, \ldots
 

\subsection{Embedding}

The \product{dyncall} library has a very low dependency to system facilities.
The library uses some heap-memory to store the Call VM and uses per default
\capi{malloc()} and \capi{free()} calls. This behaviour can be changed by 
providing custom \capi{dcAllocMem()} and \capi{dcFreeMem()} functions.
See \shell{dyncall/dyncall\_alloc.h} for details.


\subsection{Multi-threading}

The \product{dyncall} library is thread-safe and reentrant, by means that it
works correctly during execution of multiple threads if, and only if there is
at most a single thread pushing arguments to a CallVM (invoking the call is
always thread-safe, though). However, since there's no limitation on the
number of created CallVM objects, it is advised to keep a copy for each
thread.


\subsection{Supported types}

Currently, the \product{dyncall} library supports all of ANSI C's integer,
floating point and pointer types as function call arguments as well as return
values. Additionally, C++'s \capi{bool} type is supported. Due to the still
rare and often incomplete support of the \capi{long double} type on various
platforms, the latter is currently not supported.


\subsection{Roadmap}

The \product{dyncall} library should be extended by a wide variety of other
calling conventions and ported to other, more esoteric platforms. With its low
memory footprint it surely might come in handy on embedded systems. So far
dyncall supports arm32 and mips32 (eabi) embedded systems processors.
Furthermore, the authors plan to write some more scripting language bindings,
examples, and other projects that are based on it.\\
Besides \product{dyncall} and \product{dyncallback}, the \product{dynload}
library needs to be extended with support for other shared library formats
(e.g. AmigaOS .library or GEM \cite{.ldg} files).


\subsection{Related libraries}

Besides the \product{dyncall} library, there are other free and open projects
with similar goals. The most noteworthy libraries are libffi \cite{libffi},
C/Invoke \cite{cinvoke} and libffcall \cite{libffcall}.
